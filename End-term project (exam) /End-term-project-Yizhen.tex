\documentclass[11pt,]{article}
\usepackage[left=1in,top=1in,right=1in,bottom=1in]{geometry}
\newcommand*{\authorfont}{\fontfamily{phv}\selectfont}
\usepackage[]{mathpazo}


  \usepackage[T1]{fontenc}
  \usepackage[utf8]{inputenc}




\usepackage{abstract}
\renewcommand{\abstractname}{}    % clear the title
\renewcommand{\absnamepos}{empty} % originally center

\renewenvironment{abstract}
 {{%
    \setlength{\leftmargin}{0mm}
    \setlength{\rightmargin}{\leftmargin}%
  }%
  \relax}
 {\endlist}

\makeatletter
\def\@maketitle{%
  \newpage
%  \null
%  \vskip 2em%
%  \begin{center}%
  \let \footnote \thanks
    {\fontsize{18}{20}\selectfont\raggedright  \setlength{\parindent}{0pt} \@title \par}%
}
%\fi
\makeatother




\setcounter{secnumdepth}{0}

\usepackage{color}
\usepackage{fancyvrb}
\newcommand{\VerbBar}{|}
\newcommand{\VERB}{\Verb[commandchars=\\\{\}]}
\DefineVerbatimEnvironment{Highlighting}{Verbatim}{commandchars=\\\{\}}
% Add ',fontsize=\small' for more characters per line
\usepackage{framed}
\definecolor{shadecolor}{RGB}{248,248,248}
\newenvironment{Shaded}{\begin{snugshade}}{\end{snugshade}}
\newcommand{\AlertTok}[1]{\textcolor[rgb]{0.94,0.16,0.16}{#1}}
\newcommand{\AnnotationTok}[1]{\textcolor[rgb]{0.56,0.35,0.01}{\textbf{\textit{#1}}}}
\newcommand{\AttributeTok}[1]{\textcolor[rgb]{0.77,0.63,0.00}{#1}}
\newcommand{\BaseNTok}[1]{\textcolor[rgb]{0.00,0.00,0.81}{#1}}
\newcommand{\BuiltInTok}[1]{#1}
\newcommand{\CharTok}[1]{\textcolor[rgb]{0.31,0.60,0.02}{#1}}
\newcommand{\CommentTok}[1]{\textcolor[rgb]{0.56,0.35,0.01}{\textit{#1}}}
\newcommand{\CommentVarTok}[1]{\textcolor[rgb]{0.56,0.35,0.01}{\textbf{\textit{#1}}}}
\newcommand{\ConstantTok}[1]{\textcolor[rgb]{0.00,0.00,0.00}{#1}}
\newcommand{\ControlFlowTok}[1]{\textcolor[rgb]{0.13,0.29,0.53}{\textbf{#1}}}
\newcommand{\DataTypeTok}[1]{\textcolor[rgb]{0.13,0.29,0.53}{#1}}
\newcommand{\DecValTok}[1]{\textcolor[rgb]{0.00,0.00,0.81}{#1}}
\newcommand{\DocumentationTok}[1]{\textcolor[rgb]{0.56,0.35,0.01}{\textbf{\textit{#1}}}}
\newcommand{\ErrorTok}[1]{\textcolor[rgb]{0.64,0.00,0.00}{\textbf{#1}}}
\newcommand{\ExtensionTok}[1]{#1}
\newcommand{\FloatTok}[1]{\textcolor[rgb]{0.00,0.00,0.81}{#1}}
\newcommand{\FunctionTok}[1]{\textcolor[rgb]{0.00,0.00,0.00}{#1}}
\newcommand{\ImportTok}[1]{#1}
\newcommand{\InformationTok}[1]{\textcolor[rgb]{0.56,0.35,0.01}{\textbf{\textit{#1}}}}
\newcommand{\KeywordTok}[1]{\textcolor[rgb]{0.13,0.29,0.53}{\textbf{#1}}}
\newcommand{\NormalTok}[1]{#1}
\newcommand{\OperatorTok}[1]{\textcolor[rgb]{0.81,0.36,0.00}{\textbf{#1}}}
\newcommand{\OtherTok}[1]{\textcolor[rgb]{0.56,0.35,0.01}{#1}}
\newcommand{\PreprocessorTok}[1]{\textcolor[rgb]{0.56,0.35,0.01}{\textit{#1}}}
\newcommand{\RegionMarkerTok}[1]{#1}
\newcommand{\SpecialCharTok}[1]{\textcolor[rgb]{0.00,0.00,0.00}{#1}}
\newcommand{\SpecialStringTok}[1]{\textcolor[rgb]{0.31,0.60,0.02}{#1}}
\newcommand{\StringTok}[1]{\textcolor[rgb]{0.31,0.60,0.02}{#1}}
\newcommand{\VariableTok}[1]{\textcolor[rgb]{0.00,0.00,0.00}{#1}}
\newcommand{\VerbatimStringTok}[1]{\textcolor[rgb]{0.31,0.60,0.02}{#1}}
\newcommand{\WarningTok}[1]{\textcolor[rgb]{0.56,0.35,0.01}{\textbf{\textit{#1}}}}



\title{End-term project: Advanced Statistical Computing 2020  }



\author{\Large Yizhen Dai\vspace{0.05in} \newline\normalsize\emph{s2395479}  }


\date{}

\usepackage{titlesec}

\titleformat*{\section}{\normalsize\bfseries}
\titleformat*{\subsection}{\normalsize\itshape}
\titleformat*{\subsubsection}{\normalsize\itshape}
\titleformat*{\paragraph}{\normalsize\itshape}
\titleformat*{\subparagraph}{\normalsize\itshape}


\usepackage{natbib}
\bibliographystyle{apsr}
\usepackage[strings]{underscore} % protect underscores in most circumstances



\newtheorem{hypothesis}{Hypothesis}
\usepackage{setspace}


% set default figure placement to htbp
\makeatletter
\def\fps@figure{htbp}
\makeatother

\usepackage{hyperref}

% move the hyperref stuff down here, after header-includes, to allow for - \usepackage{hyperref}

\makeatletter
\@ifpackageloaded{hyperref}{}{%
\ifxetex
  \PassOptionsToPackage{hyphens}{url}\usepackage[setpagesize=false, % page size defined by xetex
              unicode=false, % unicode breaks when used with xetex
              xetex]{hyperref}
\else
  \PassOptionsToPackage{hyphens}{url}\usepackage[draft,unicode=true]{hyperref}
\fi
}

\@ifpackageloaded{color}{
    \PassOptionsToPackage{usenames,dvipsnames}{color}
}{%
    \usepackage[usenames,dvipsnames]{color}
}
\makeatother
\hypersetup{breaklinks=true,
            bookmarks=true,
            pdfauthor={Yizhen Dai (s2395479)},
             pdfkeywords = {pandoc, r markdown, knitr},  
            pdftitle={End-term project: Advanced Statistical Computing 2020},
            colorlinks=true,
            citecolor=blue,
            urlcolor=blue,
            linkcolor=magenta,
            pdfborder={0 0 0}}
\urlstyle{same}  % don't use monospace font for urls

% Add an option for endnotes. -----


% add tightlist ----------
\providecommand{\tightlist}{%
\setlength{\itemsep}{0pt}\setlength{\parskip}{0pt}}

% add some other packages ----------

% \usepackage{multicol}
% This should regulate where figures float
% See: https://tex.stackexchange.com/questions/2275/keeping-tables-figures-close-to-where-they-are-mentioned
\usepackage[section]{placeins}


\begin{document}
	
% \pagenumbering{arabic}% resets `page` counter to 1 
%
% \maketitle

{% \usefont{T1}{pnc}{m}{n}
\setlength{\parindent}{0pt}
\thispagestyle{plain}
{\fontsize{18}{20}\selectfont\raggedright 
\maketitle  % title \par  

}

{
   \vskip 13.5pt\relax \normalsize\fontsize{11}{12} 
\textbf{\authorfont Yizhen Dai} \hskip 15pt \emph{\small s2395479}   

}

}








\begin{abstract}

    \hbox{\vrule height .2pt width 39.14pc}

    \vskip 8.5pt % \small 

\noindent This document provides an introduction to R Markdown, argues for its
benefits, and presents a sample manuscript template intended for an
academic audience. I include basic syntax to R Markdown and a minimal
working example of how the analysis itself can be conducted within R
with the \texttt{knitr} package.


\vskip 8.5pt \noindent \emph{Keywords}: pandoc, r markdown, knitr \par

    \hbox{\vrule height .2pt width 39.14pc}



\end{abstract}


\vskip -8.5pt


 % removetitleabstract

\noindent  

\hypertarget{introduction}{%
\section{Introduction}\label{introduction}}

This project aims at solving a modeling problem faced by an insurance
company - ANV. Two of ANV business lines,
\href{https://en.wikipedia.org/wiki/Professional_liability_insurance}{Professional
liability insurance} (PLI) and
\href{https://en.wikipedia.org/wiki/Workers\%27_compensation}{Workers'
compensation} (WC), were affected by a huge claim from one client during
the last year. Therefore, ANV comes to a reinsurance company for an
insurance policy. For some threshold \(t = 100,110,…,200\): - If
\(PLI+WC<=t\), ANV pays the claim themselves. - If not, the reinsurance
company pays the claim.

The price that reinsurance company asks depends on the threshold \(t\):

\[P(t)=40000 * e^{-t/7}\] Of course, the policy is only reasonable if
the expected over-threshold claim exceeds the price:
\(V(t)=E[(PLI+WC)1(PLI+WC>t)] > P(t)\). We will use statistical modeling
to approximate \(V(t)\)

\hypertarget{methodolgy}{%
\section{Methodolgy}\label{methodolgy}}

\hypertarget{simulation-study}{%
\section{Simulation study}\label{simulation-study}}

\hypertarget{results}{%
\section{Results}\label{results}}

The lion's share of a R Markdown document will be raw text, though the
front matter may be the most important part of the document. R Markdown
uses \href{http://www.yaml.org/}{YAML} for its metadata and the fields
differ from
\href{http://svmiller.com/blog/2015/02/moving-from-beamer-to-r-markdown/}{what
an author would use for a Beamer presentation}. I provide a sample YAML
metadata largely taken from this exact document and explain it below.

\hypertarget{markdown-syntax}{%
\section{Markdown Syntax}\label{markdown-syntax}}

\begin{Shaded}
\begin{Highlighting}[]

\FunctionTok{# Introduction}

\NormalTok{**Lorem ipsum** dolor *sit amet*. }

\NormalTok{- }\StringTok{Single asterisks italicize text *like this*. }
\StringTok{- Double asterisks embolden text **like this**.}

\NormalTok{Start a new paragraph with a blank line separating paragraphs.}

\NormalTok{- }\StringTok{This will start an unordered list environment, and this will be the first item.}
\StringTok{- This will be a second item.}
\StringTok{- A third item.}
\StringTok{    - Four spaces and a dash create a sublist and this item in it.}
\StringTok{- The fourth item.}
\StringTok{    }
\StringTok{1. This starts a numerical list.}
\StringTok{2. This is no. 2 in the numerical list.}
\StringTok{    }
\StringTok{# This Starts A New Section}
\StringTok{## This is a Subsection}
\StringTok{### This is a Subsubsection}
\StringTok{#### This starts a Paragraph Block.}

\NormalTok{>}\DataTypeTok{ This will create a block quote, if you want one.}

\NormalTok{Want a table? This will create one.}

\NormalTok{Table Header  | Second Header}
\NormalTok{------------- | -------------}
\NormalTok{Table Cell    | Cell 2}
\NormalTok{Cell 3        | Cell 4 }

\NormalTok{Note that the separators *do not* have to be aligned.}

\NormalTok{Want an image? This will do it.}

\AlertTok{![caption for my image](path/to/image.jpg)}

\BaseNTok{`fig_caption: yes`}\NormalTok{ will provide a caption. Put that in the YAML metadata.}

\NormalTok{Almost forgot about creating a footnote.}\OtherTok{[^1]}\NormalTok{ This will do it again.}\OtherTok{[^2]}

\OtherTok{[^1]}\NormalTok{: The first footnote}
\OtherTok{[^2]}\NormalTok{: The second footnote}

\NormalTok{Want to cite something? }

\NormalTok{- }\StringTok{Find your biblatexkey in your bib file.}
\StringTok{- Put an @ before it, like @smith1984, or whatever it is.}
\StringTok{- @smith1984 creates an in-text citation (e.g. Smith (1984) says...)}
\StringTok{- [@smith1984] creates a parenthetical citation (Smith, 1984)}

\NormalTok{That'll also automatically create a reference list at the end of the document.}

\OtherTok{[In-text link to Google](http://google.com)}\NormalTok{ as well.}
\end{Highlighting}
\end{Shaded}

That's honestly it. Markdown takes the chore of markup from your
manuscript (hence: ``Markdown'').

On that note, you could easily pass most LaTeX code through Markdown if
you're writing a LaTeX document. However, you don't need to do this
(unless you're using the math environment) and probably shouldn't anyway
if you intend to share your document in HTML as well.

\hypertarget{using-r-markdown-with-knitr}{%
\section{Using R Markdown with
Knitr}\label{using-r-markdown-with-knitr}}

\texttt{eval=FALSE} option just displays the R code (and does not run
it), \texttt{tidy=TRUE} wraps long code so it does not run off the page.
\texttt{include=FALSE} hide code and output from document.
\texttt{echo=FALSE} hides only the code, but not the output.

\begin{Shaded}
\begin{Highlighting}[]
\KeywordTok{library}\NormalTok{(stevemisc)}
\KeywordTok{data}\NormalTok{(uniondensity)}

\NormalTok{M1 <-}\StringTok{ }\KeywordTok{lm}\NormalTok{(union }\OperatorTok{~}\StringTok{ }\NormalTok{left }\OperatorTok{+}\StringTok{ }\NormalTok{size }\OperatorTok{+}\StringTok{ }\NormalTok{concen, }\DataTypeTok{data =}\NormalTok{ uniondensity)}

\KeywordTok{library}\NormalTok{(arm)}
\KeywordTok{coefplot}\NormalTok{(M1)}
\end{Highlighting}
\end{Shaded}

The implications for workflow are faily substantial. Authors can rather
quickly display the code they used to run the analyses in the document
itself (likely in the appendix). As such, there's little guesswork for
reviewers and editors in understanding what the author did in the
analyses reported in the manuscript.

\hypertarget{figure}{%
\subsection{Figure}\label{figure}}

\texttt{eval=FALSE} changes to \texttt{eval=TRUE}

\begin{Shaded}
\begin{Highlighting}[]
\KeywordTok{coefplot}\NormalTok{(M1)}
\end{Highlighting}
\end{Shaded}

\hypertarget{table}{%
\subsection{Table}\label{table}}

add \texttt{results="asis"} to the brackets to start the R code chunk.
The ensuing output will look like this (though the table may come on the
next page).

\begin{Shaded}
\begin{Highlighting}[]
\KeywordTok{stargazer}\NormalTok{(M1, }\DataTypeTok{title =} \StringTok{"A Handsome Table"}\NormalTok{, }\DataTypeTok{header =} \OtherTok{FALSE}\NormalTok{)}
\end{Highlighting}
\end{Shaded}

\hypertarget{footnote}{%
\subsection{Footnote}\label{footnote}}

Adding \texttt{echo="FALSE"} inside the brackets to start the R chunk
will omit the presentation of the R commands. It will just present the
table. This provides substantial opportunity for authors in doing their
analyses. Now, the analysis and presentation in the form of a polished
manuscript can be effectively simultaneous.\footnote{I'm not sure if I'm
  ready to commit to this myself since my workflow is still largely
  derived from
  \href{http://robjhyndman.com/hyndsight/workflow-in-r/}{Rob J.
  Hyndman's example}. However, \emph{knitr} has endless potential,
  especially when analyses can stored in cache, saved as chunks, or
  loaded in the preamble of a document to reference later in the
  manuscript.}

For my template, I'm pretty sure this is mandatory.\footnote{The main
  reason I still use \texttt{pdflatex} (and most readers probably do as
  well) is because of LaTeX fonts.
  \href{http://www-rohan.sdsu.edu/~aty/bibliog/latex/gripe.html}{Unlike
  others}, I find standard LaTeX fonts to be appealing.}

\hypertarget{how-to-cite}{%
\subsection{How to cite}\label{how-to-cite}}

Perhaps the greatest intrigue of R Markdown comes with the
\href{http://yihui.name/knitr/}{\texttt{knitr} package} provided by
\citet{xie2013ddrk}.





\newpage
\singlespacing 
\bibliography{master.bib}

\end{document}
